\documentclass[11pt]{homework}
\usepackage{quiver}

\declaretheoremstyle[
    headindent=1em,
    headfont=\color{maincolor}\normalfont\bfseries,
]{indented}

\theoremstyle{indented}
\newtheorem{axiom}{Axiom}

\title{Chern Class}
\author{Tao Huang}
\email{me@tau.ovh}

\begin{document}
    
\maketitle

% All cohomology groups are with integer coefficients without notation. 

\section{Via an Euler class}
\begin{theorem}[Thom isomorphism]
    Let $\pi:E\to B$ be an oriented rank $n$ real vector bundle. Then the cohomology group $H^{i}\left(E, E_{0}\right)$ is zero for $i<n$, and $H^{n}\left(E, E_{0}\right)$ contains one and only one cohomology class $u$ whose restriction
    \begin{equation*}
        u \vert_{\left(F, F_{0}\right)} \in H^{n}\left(F, F_{0}\right)
    \end{equation*}
    is equal to the preferred generator $u_{F}$ for every fiber $F$. Furthermore the correspondence $y \mapsto y \smile u$ maps $H^{k}(E)$ isomorphically onto $H^{k+n}\left(E, E_{0}\right)$ for every integer $k$.
\end{theorem} 
\begin{definition}
    The inclusion $(E,\emptyset )\hookrightarrow (E, E_{0})$ induce maps
    \begin{equation*}
        H^{r}(E,E_{0})\to H^{r}(E) \cong H^{r}(B).
    \end{equation*}
    The \textit{Euler class} $e(E) \in H^n(B)$ is the image of $u$ under the composition of these maps.
\end{definition}
Note that since $E$ and $B$ are homotopic, $H^{k}(E) \cong H^{k}(B)$.
\begin{proposition}[Gysin sequence]
    \label{prop:gysin}
    Ther is a long exact sequence:
    \begin{equation*}
        \cdots \to H^{k+n-1}(E_0) \to H^{k}(B) \xrightarrow{\smile e} H^{k+n}(B) \xrightarrow{\pi_0^*} H^{k+n}(E_0) \to \cdots.
    \end{equation*}
\end{proposition}
\begin{proof}
    By long exact sequence:
    \begin{equation*}
        \cdots \to H^{r-1}(E_0) \to H^{r}(E, E_0) \to H^{r}(E) \to H^{r}(E_0) \to \cdots.
    \end{equation*}
    Then the following diagram commutes:
    \[\begin{tikzcd}
        {H^{k+n}(E,E_0)} && {H^{k+n}(E)} && {H^{k+n}(E_0)} \\
        {H^k(E)} && {} \\
        {H^k(B)} && {H^{k+n}(B)}
        \arrow["{\pi_0^*}"', from=3-3, to=1-5]
        \arrow[from=1-3, to=1-5]
        \arrow["{\pi^*}", from=3-3, to=1-3]
        \arrow["{\smile e(E)}", from=3-1, to=3-3]
        \arrow[from=1-1, to=1-3]
        \arrow["\cong"', from=2-1, to=3-1]
        \arrow["\cong"', from=1-1, to=2-1]
    \end{tikzcd}\]
\end{proof}

\begin{corollary}
    Let $\pi : E\to B$ be a rank $n$ complex vector bundle, there is an isomorphism:
    \begin{equation*}
        H^{k}(B) \xrightarrow[\cong]{\pi_0^*} H^{k}(E_0),
    \end{equation*}
    for $k<2n-1$.
\end{corollary}
\begin{proof}
    $H^{k-2n}(B) \cong H^{k-2n+1}(B) \cong 0$. Together with proposition \ref{prop:gysin}.
\end{proof}

\begin{definition}
    Let $\pi : E\to B$ be a rank $n$ complex vector bundle over a paracompact space $B$. \textit{Chern class} $c_{k}(E) \in H^{2k}(B_{\mathbb {R}})$ is given by
    \begin{equation*}
        c_{k}(E)=\begin{cases}
            {\pi_0^{*}}^{-1}c_{k}(E')&k<n\\
            e(E_{\mathbb {R} })&k=n\\
            0&k>n
        \end{cases},
    \end{equation*}
    where $E' \to E_0$ is a vector bundle whose fiber on each point $v\in E_0$ is the quotient space $E / \mathbb{C}\{v\}$, whence $E'$ is a rank $n-1$ complex vector bundle.
\end{definition}


\section{via Chern–Weil theory}
\subsection{Curvature from}
Let $\pi: E\to M$ be a smooth rank $n$ complex vector bundle over a smooth manifold $M$. Denote the space of smooth sections of $E$ over $M$ by $\Gamma(M,E)$. 

\begin{definition}
    We call any section of $\wedge^k T^*M \otimes E$ an \textit{$E$-valued $k$-form} on $M$. The set of $E$-valued $k$-forms $\Gamma (M, \wedge^k T^{*}M\otimes E)$ is denoted by $\Omega^k(M;E)$
\end{definition}

\begin{definition}
    A \textit{connection} on E is an $\mathbb {C}$-linear map
    \begin{equation*}
        \nabla :\Omega^0(M;E)\to \Omega^1(M;E)
    \end{equation*}
    such that the Leibniz rule
    \begin{equation*}
        \nabla (fs)=\mathrm{d}f\otimes s+f\nabla s
    \end{equation*}
    holds for all smooth functions $f$ on $M$ and all smooth sections $s$ of $E$.

    Let $X$ be a tangent vector field on $M$, one can define a \textit{covariant derivative along} $X$
    \begin{equation*}
        \nabla _{X}:\Omega^{0}(M;E)\to \Omega^{0}(M;E),
    \end{equation*}
    by contracting $X$ with the resulting covariant index in the connection: $\nabla _{X}(s)=(\nabla (s))(X)$. 
\end{definition}

\begin{definition}
    Let $\varphi_U: U \times \mathbb{C}^n \to \pi^{-1}(U)$ be local trivializations of $E$. A set of local sections $\{s_1, s_2, \cdots, s_n\}$ is said to be basis of $\Gamma(U,\pi^{-1}(U))$ if for all point $p \in U, \{s_1(p), s_2(p), \cdots, s_n(p)\}$ is a basis for the fiber $E_p$. Such $\{s_1, s_2, \cdots, s_n\}$ is called \textit{local frame} of $E$ over $U$.
\end{definition}

\begin{definition}
    For the given connection $\nabla$ and local frame $\{s_1, s_2, \cdots, s_n\}$ on $U$, we can write:
    \begin{equation*}
        \nabla_X s_i = \sum_j \left(\omega_U\right)_{i,j}(X) s_j,
    \end{equation*}
    for all vector field $X$ over $U$, where $\omega_U \in \Omega^1(M;\mathfrak{gl}(n;\mathbb{C}))$, called \textit{connection $1$-form} associated to the given local frame.
\end{definition}
 
\begin{definition}
    Let $\nabla$ be the given connection on $E$, one can extend $\nabla$ to a family of operators:
    \begin{equation*}
        \nabla : \Omega^{k}(M;E)\to \Omega^{k+1}(M;E),
    \end{equation*}
    by defining
    \begin{equation*}
        \nabla (\omega \otimes s) = \mathrm{d} \omega \otimes s + (-1)^k \omega \wedge \nabla s,
    \end{equation*}
    for all $\omega \in \Omega^{k}(M), s\in \Omega^{0}(M;E)$
\end{definition}

\begin{definition}
    We call $R_\nabla = \nabla^2: \Omega^0(M;E)\to \Omega^2(M;E)$ the \textit{curvature tensor} of the connection $\nabla$. In local frame:
    \begin{equation*}
        \nabla^2 s_i (X,Y) = \sum_j \left(\Omega_U\right)_{i,j}(X, Y) s_j,
    \end{equation*}
    where $\Omega_U \in \Omega^2(M;\mathfrak{gl}(n;\mathbb{C}))$ is called \textit{curvature 2-form}.
\end{definition}

\subsection{Invariant polynomial}

\begin{proposition}
    For the given vector space $V$, there is a bijection $\mathrm{Sym}^k V^*$ and homogeneous polynomial of degree $k$:
    \begin{equation*}
        T \xmapsto{\cong} P_T := \left(v \mapsto T(v,\cdots,v)\right).
    \end{equation*}
\end{proposition}
\begin{proof}
    Just notice the \textit{polarization formula}:
    \begin{equation*}
        T(v_1, \cdots, v_k) = \frac{1}{k!}\frac{\partial^k}{\partial t_1 \cdots \partial t_k} P_T(t_1 v_1 + \cdots + t_k v_k).
    \end{equation*}
\end{proof}

In general, let $G = GL(n;\mathbb{C})$ and $\mathfrak{g} = \mathfrak{gl} (n;\mathbb{C})$.

\begin{definition}
    A symmetric $k$-tensor $T \in \mathrm{Sym}^k \mathfrak g^*$ is called \textit{$G$-invariant} if 
    \begin{equation*}
        g \cdot T = T \quad \forall g \in G.
    \end{equation*}
    The set of all $G$-invariant symmetric $k$-tensor is denoted by $I^k(G)$.
\end{definition}

\begin{theorem}[Chevally restriction theorem]
    Let $G$ be a complex semi-simple Lie group, $\mathfrak{g}$ the corresponding Lie algebra, $\mathfrak{h}$ the Cartan subalgebra and $W$ the Weyl group of $\mathfrak{g}$. 
    Then $G$-invariant polynomial on $\mathfrak{g}$ is isomorphic to $W$-invariant polynomial on $\mathfrak{h}$. 
    % Denote $\mathbb{C} [\mathfrak{g}]^G$ the $G$-invariant polynomial on $\mathfrak{g}$ and $\mathbb{C} [\mathfrak{h}]^W$ the $W$-invariant polynomial on $\mathfrak{h}$. Then
    % \begin{equation*}
    %     \mathbb{C} [\mathfrak{g}]^G \cong \mathbb{C} [\mathfrak{h}]^W.
    % \end{equation*} 
\end{theorem}

\subsection{Chern-Weil theory}

\begin{theorem}
    \label{thm:chern_weil}
    Let $E$ be a vector bundle over $M$. Then
    \begin{enumerate}
        \item For any $T \in I^{k}(G)$ and any linear connection $\nabla$ on $E, P_{T}\left(\Omega\right)$ is a closed $2 k$-form.
        \item The de Rham cohomology class $\left[P_{T}\left(\Omega\right)\right] \in H_{d R}^{2 k}(M)$ is independent of the choices of $\nabla$.
        \item The Chern-Weil maps
        \begin{equation*}
            \mathcal{C W}:\left(I^{*}(G), \circ\right) \to\left(H_{dR}^{*}(M), \wedge\right), \quad T \mapsto\left[P_{T}\left(\Omega\right)\right]
        \end{equation*}
        is a ring homomorphism, which is called \textit{Chern-Weil homomorphism}.
    \end{enumerate}
\end{theorem}

\begin{lemma}[Bianchi identity]
    \begin{equation*}
        \mathrm d \, \Omega = \Omega \wedge \omega - \omega \wedge \Omega = [\Omega, \omega].
    \end{equation*}
\end{lemma}
\begin{lemma}
    For all $T \in I^{k}(G)$ and $X, X_1, \cdots, X_k \in \mathfrak{g}$, we have
    \begin{equation*}
        T([X,X_1], X_2, \cdots, X_k ) + T(X_1, [X,X_2], \cdots, X_k ) + T(X_1, X_2, \cdots, [X, X_k]) = 0.
    \end{equation*}
\end{lemma}

\begin{proof}[proof of theorem \ref{thm:chern_weil}]
    \begin{enumerate}
        \item 
        \begin{align*}
            \mathrm d \, P_{T}\left(\Omega\right) &= \mathrm d \, T(\Omega, \cdots, \Omega) \\
            &= \sum T(\Omega, \cdots, \mathrm d \, \Omega, \cdots, \Omega) \\
            &= \sum T(\Omega, \cdots, \left[\Omega, \omega \right], \cdots, \Omega) \\
            &=0.
        \end{align*}
        \item 
        Let $\nabla_1, \nabla_2$ be two different connection on $E$. The mapping $p_1 : M \times \mathbb{R} \to M$ induces two connection $\tilde \nabla_1, \tilde \nabla_2$ on $p_1^\ast E$.
        \begin{equation*}
            \tilde \nabla = s \tilde \nabla_1 + (1-s) \tilde \nabla_2
        \end{equation*}
        is also a connection on $p_1^\ast E$.
        
        Consider $i_\epsilon: x\mapsto (x, \epsilon)$, then $i_0^\ast \tilde \nabla = \nabla_0, i_1^\ast \tilde \nabla = \nabla_1$. Hence
        \begin{equation*}
            i_j^\ast \left[P_T\left(\tilde \Omega\right)\right] = \left[P_T\left(\Omega_j\right)\right], \quad j=1,2.
        \end{equation*}
        Since the mapping $i_0$ is homotopic to $i_1$, it follows that
        \begin{equation*}
            \left[P_T\left(\Omega_0\right)\right] = \left[P_T\left(\Omega_1\right)\right].
        \end{equation*}
        \item For all $T\in I^{k}(G), S \in I^{l}(G), X_1, \cdots, X_{k+l} \in \Omega^\ast (M)$,
        \begin{equation*}
            P_{T\circ S}(X_1, \cdots, X_{k+l}) = \frac{1}{(k+l)!} \sum_{\sigma \in \mathfrak{S}_{k+l}} T(X_{\sigma(1)}, \cdots, X_{\sigma(k)}) \wedge S(X_{\sigma(k+1)}, \cdots, X_{\sigma(k+l)}).
        \end{equation*}
    \end{enumerate}
\end{proof}

\subsection{Chern Class}

\begin{definition}
    For the given $G$-invariant polynomial $f$ of degree $k$, since the cohomology class $\left[f(\Omega)\right] \in H_{dR}^{2k}(M)$ does not depend on connections $\Omega$, we will denote it by $f(E)$ and call it the \textit{characteristic class} of $E$ corresponding to $f$. The \textit{total Chern class} $c(E)$ is defined by:
    \begin{equation*}
        \left[\det \left(I - \frac{1}{2\pi i}\Omega\right)\right] = 1 + \sum_{k=1}^n \left(\frac{-1}{2 \pi i}\right)^k \sigma_k (\Omega).
    \end{equation*}
\end{definition}

\begin{remark}
    \begin{equation*}
        H^{k} \left(\mathbb{CP}^n\right) = \begin{cases}
            \mathbb{R},& \text{for } k \text{ even and } 0\leq k \leq 2n\\
            0,              & \text{otherwise}
        \end{cases}.
    \end{equation*}
    Moreover, $H^{k} \left(\mathbb{CP}^n\right) = \mathbb{R}[a] / \left(a^{n+1}\right)$, where $a$ is the generator of $H^2\left(\mathbb{CP}^k; \mathbb{Z}\right)$.
\end{remark}

\begin{proof}
    Denote $U = \mathbb{CP}^n - \mathbb{CP}^{n-1} \cong \mathbb{C}^n$ and $V = \mathbb{CP}^n \backslash \{[0:\cdots:0:1]\}$. $U \cap V = \mathbb{C}^n - \{0\} \approx \mathbb{S}^{2n-1}$. Note that
    \begin{equation*}
        ([z_0:\cdots:z_{n}], t) \mapsto [z_0: \cdots:z_{n-1}: t z_n]
    \end{equation*}
    defines a homotopy between $\mathbb{CP}^{n-1}$ and $V$.
    Using Mayer–Vietoris sequence, the result is clear:
    \begin{align*}
        \cdots \to H^{k-1} \left( \mathbb{S}^{2n-1} \right) \to H^{k} \left(\mathbb{CP}^n \right) \to H^{k} \left( \mathbb{CP}^{n-1} \right) \to \cdots.
    \end{align*}
    % \begin{equation*}
    %     H^\ast \left(\mathbb{CP}^{n-1}\right) = H^\ast \left(\mathbb{CP}^n\right) + H^\ast \left(\mathbb{S}^{2n-1}\right).
    % \end{equation*}
\end{proof}

\begin{example}
    Riemann sphere, which is isomorphic to complex projective line $\mathbb{CP}^1$, has Kahler metric:
    \begin{equation*}
        \frac{\mathrm d z \, \mathrm d \bar{z}}{\left(1+|z|^2\right)^2}
    \end{equation*}
    Hence there is a connection on tangent bundle $T \mathbb{CP}^1$ induced by the metric:
    \begin{align*}
        2\frac{z \, \mathrm d \, \bar z + \bar z \, \mathrm \, d z}{\left(1+|z|^2\right)^3} &= \mathrm{d}\, \frac{1}{\left(1+|z|^2\right)^2} \\
        &= \mathrm{d}\, \left\langle \frac{\partial}{\partial z}, \frac{\partial}{\partial \bar z} \right\rangle \\
        &= \left\langle \nabla \frac{\partial}{\partial z}, \frac{\partial}{\partial \bar z} \right\rangle + \left\langle \frac{\partial}{\partial z}, \nabla \frac{\partial}{\partial \bar z} \right\rangle
    \end{align*}
    \begin{equation*}
        \nabla \frac{\partial}{\partial z} = \frac{2z \, \mathrm d \bar z}{1+|z|^2} \otimes \frac{\partial}{\partial z}.
    \end{equation*}
    Then it has curvature form 
    \begin{equation*}
        \Omega = \frac{2\mathrm d z \wedge \mathrm d \bar z}{\left(1+|z|^2\right)^2}.
    \end{equation*}
    To see the Chern class $c_1(T \mathbb{CP}^1) = [\frac{i}{2\pi}\mathrm{tr} \,\Omega] = [\frac{i}{2\pi}\Omega] \in H^2_{dR}\left(\mathbb{CP}^1\right)$ is non-zero, 
    \begin{equation*}
        \int c_1 = \frac{i}{\pi} \int \frac{\mathrm d z \wedge \mathrm d \bar z}{\left(1+|z|^2\right)^2} = 2.
    \end{equation*}
\end{example}

\begin{proposition}
    The tirvial bundle $E = M\otimes \mathbb{C}^n$ has total Chern class $c(E) = 1$.
\end{proposition}
\begin{proof}
    Sections $s$ of $E$ can be written by $s = (f_1, \cdots, f_n)$ and $E$ has tirvial connection:
    \begin{equation*}
        \nabla (f_1, \cdots, f_n) = (\mathrm{d}\, f_1, \cdots, \mathrm{d}\, f_n),
    \end{equation*}
    since
    \begin{align*}
        \nabla f\cdot (f_1, \cdots, f_n) &= \nabla (f\cdot f_1, \cdots, f\cdot f_n) \\
        &= (\mathrm{d}\, f\cdot f_1 + f\cdot  \mathrm{d}\, f_1, \cdots, \mathrm{d}\, f\cdot f_n +f\cdot \mathrm{d}\, f_n) \\
        &= \mathrm{d}\, f \otimes (f_1, \cdots, f_n) + f \nabla (f_1, \cdots, f_n).
    \end{align*}

    Then 
    \begin{align*}
        \nabla^2 (f_1, \cdots, f_n) &= \nabla (\mathrm{d}\, f_1, \cdots, \mathrm{d}\, f_n) \\
        &= (\mathrm{d}^2\, f_1, \cdots, \mathrm{d}^2\, f_n) + (\mathrm{d}^2\, f_1, \cdots, \mathrm{d}^2\, f_n) \\
        &= 0.
    \end{align*}

    Hence the curvature 2-form $\Omega = 0$.
\end{proof}

\section{Axiomatic definition}
\begin{definition}
    The Chern classes satisfy the following four axioms:
        \begin{axiom} $c_{0}(E)=1$ for all $E$.\end{axiom}

        \begin{axiom}[Naturality] If $f : Y \to X$ is continuous and $f^*E$ is the vector bundle pullback of $E$, then $(f^{*}E)=f^{*}c_{k}(E)$.\end{axiom}

        \begin{axiom}[Whitney sum formula]
            \label{axm:whitney_sum}
            If $F\to X$ is another complex vector bundle, then the Chern classes of the direct sum $E\oplus F$ are given by
            \begin{equation*}
                c(E\oplus F)=c(E)\smile c(F).
            \end{equation*}
        \end{axiom}
        \begin{axiom}[Normalization] The total Chern class of the tautological line bundle over $\mathbb{CP}^k$ is $1-a$, where $a$ is Poincare-dual to the hyperplane $\mathbb{CP}^{k-1}\subseteq \mathbb{CP}^k$, i.e. the generator of $H^2\left(\mathbb{CP}^k; \mathbb{Z}\right)$ \end{axiom}
\end{definition}

\begin{theorem}[Poincare duality]
    If a manifold $M$ is orientable and compact, then 
    \begin{align*}
        I_M : H^{k}(M) &\to \left(H^{n-k}(M)\right)^\ast \\
        \omega &\mapsto \left( \eta \mapsto \int_M \omega \wedge \eta \right)
    \end{align*}
    is an isomorphism.
\end{theorem}

\begin{remark}
    For the complex projective space $\mathbb{CP}^n$, the hyperplane bundle 
    \begin{equation*}
        \mathcal{O}_{\mathbb{CP}^n}(1) = \left\{(V,\lambda) \mid V \in \mathbb{CP}^n , \lambda \in V^\ast \right\}
    \end{equation*}
    has the property that there is a exact sequence:
    \begin{equation*}
        0 \to \mathbb{CP}^n \otimes \mathbb C  \to \mathcal{O}_{\mathbb{CP}^n}(1) ^{\oplus (n+1)} \to T \mathbb{CP}^n \to 0,
    \end{equation*}
    since
    \begin{align*}
        T \mathbb{CP}^n \oplus \left(\mathbb{CP}^n \otimes \mathbb C\right) &= \mathrm{Hom}(\mathcal{O}(-1), \eta) \oplus \mathrm{Hom}(\mathcal{O}(-1), \mathcal{O}(-1)) \\
        &= \mathrm{Hom}(\mathcal{O}(-1), \mathbb{CP}^n \otimes \mathbb C) \\
        &= \mathcal{O}_{\mathbb{CP}^n}(1) ^{\oplus (n+1)},
    \end{align*}
    where $\eta = \left\{(V,v) \mid V \in \mathbb{CP}^n , v \in V^{\bot} \right\}$.
    Note that $\phi \in \mathrm{Hom}(\mathcal{O}(-1), \eta)$ defines an element of $T_V \mathbb{CP}^n$ by 
    \begin{equation*}
        \mathrm{Id} \oplus \phi : V \to V \oplus V^{\bot} = \mathbb{C}^{n+1}.
    \end{equation*}

    Denote $a = c\left(\mathcal{O}_{\mathbb{CP}^n}(1)\right)$Note that $c\left(\mathbb{CP}^n \otimes \mathbb C\right) =1$, whence 
    \begin{equation*}
        c \left(T \mathbb{CP}^n\right) = c\left(\mathcal{O}_{\mathbb{CP}^n}(1) ^{\oplus (n+1)}\right) = (1+a)^{n+1}.
    \end{equation*}
\end{remark}

\begin{remark}
    In local affine coordinates, complex projective space $\mathbb{CP}^n$ has Fubini–Study metric:
    \begin{equation*}
        ds^{2} = \frac {(1+z_{i}{\bar {z}}^{i})\,dz_{j}\,d{\bar {z}}^{j}-{\bar {z}}^{j}z_{i}\,dz_{j}\,d{\bar {z}}^{i}}{(1+z_{i}{\bar {z}}^{i})^{2}}.
    \end{equation*}
\end{remark}

\section{Application}

\subsection{Chern–Gauss–Bonnet theorem}

\begin{theorem}[Fundamental Theorem of Riemannian Geometry]
    Let $(M, g)$ be a Riemannian manifold (or pseudo-Riemannian manifold). Then there is a unique connection $\nabla$ called \textit{Levi-Civita connection} on the tangent bundle of $M$ which satisfies the following conditions:
    \begin{enumerate}
        \item (compatible with the metric)
        for any vector fields $X, Y, Z$ we have
        \begin{equation*}
            \partial _{X}\langle Y,Z\rangle =\langle \nabla _{X}Y,Z\rangle +\langle Y,\nabla _{X}Z\rangle ,
        \end{equation*}
        where $\partial _{X}\langle Y,Z\rangle$ denotes the derivative of the function $\langle Y,Z\rangle$ along vector field X.
        \item (torsion free)
        for any vector fields X, Y,
        \begin{equation*}
            \nabla _{X}Y-\nabla _{Y}X=[X,Y]
        \end{equation*}
        where $[X, Y]$ denotes the Lie bracket for vector fields $X, Y$.
    \end{enumerate}
\end{theorem}

\begin{theorem}[Chern–Gauss–Bonnet]
    For a closed even-dimensional Riemannian manifold $M$.
    \begin{equation*}
        \chi (M)=\int _{M}e(\Omega ),
    \end{equation*}
    where $\chi (M)$ denotes the Euler characteristic of $M$, and $\Omega$ is the associated curvature form of the Levi-Civita connection.
\end{theorem}

\subsection{Chern polynomial}

\begin{definition}
    For a complex vector bundle E, the \textit{Chern polynomial} of $E$ is given by:
    \begin{equation*}
        c_{t}(E)=1+c_{1}(E)t+\cdots +c_{n}(E)t^{n}.
    \end{equation*}

    Now, if $E=L_{1}\oplus \cdots \oplus L_{n}$ is a direct sum of complex line bundles, then it follows from the Whitney sum formula (Axiom \ref{axm:whitney_sum}) that:
    \begin{equation*}
        c_{t}(E)=(1+a_{1}(E)t)\cdots (1+a_{n}(E)t),
    \end{equation*}
    where $a_{i}(E)=c_{1}(L_{i})$ are the first Chern classes.

    The roots $a_{i}(E)$, called the \textit{Chern roots} of $E$, determine the coefficients of the polynomial:
    \begin{equation*}
        c_{k}(E)=\sigma _{k}(a_{1}(E),\ldots ,a_{n}(E)).
    \end{equation*}
\end{definition}

\end{document}